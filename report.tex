\documentclass[conference]{IEEEtran}
\IEEEoverridecommandlockouts
% The preceding line is only needed to identify funding in the first footnote. If that is unneeded, please comment it out.
\usepackage{cite}
\usepackage{amsmath,amssymb,amsfonts}
\usepackage{algorithmic}
\usepackage{graphicx}
\usepackage{textcomp}
\usepackage[T5]{fontenc}
\usepackage[utf8]{inputenc}
\usepackage[vietnamese]{babel}
\usepackage{xcolor}
\usepackage{siunitx}
\def\BibTeX{{\rm B\kern-.05em{\sc i\kern-.025em b}\kern-.08em
    T\kern-.1667em\lower.7ex\hbox{E}\kern-.125emX}}
\begin{document}

\title{CAFA 6 Protein Function Prediction Report}

\author{\IEEEauthorblockN{Trung Hau Tran, Trung Hieu Pham, Nam Khanh Pham}
\IEEEauthorblockA{\textit{VNU University of Engineering and Technology}\\
Hanoi, Vietnam}}


\maketitle

\begin{abstract}
Bài báo cáo này trình bày phương pháp của chúng tôi cho CAFA 6 Protein Function Prediction.
Cuộc thi này tập trung vào việc xây dựng mô hình học máy có khả năng dự đoán được chức năng
sinh học của protein chỉ dựa trên trình tự axit amin của các protein đó. Dữ liệu gồm các chuỗi protein 
và trình tự axit amin tương ứng, cùng với các nhãn chức năng sinh học được gán với Gene Ontology (GO).

Chúng tôi áp dụng phương pháp học máy và học sâu, kết hợp các kỹ thuật trích xuất đặc trưng 
, lựa chọn mô hình embedding phù hợp và kĩ thuật ensemble mô hình để xây dựng mô hình dự đoán chức năng protein.

Điểm cuối cùng đạt được là 0.358, top 56/460 đội (tính đến ngày 9/8/2025).
\end{abstract}

% \begin{IEEEkeywords}
% Ariel Data Challenge 2025, Signal Processing, Polynomial Interpolation, Machine Learning,
% Spectral Recovery, Time Series Analysis, Calibration Correction
% \end{IEEEkeywords}

\section{Giới thiệu}

Câu hỏi “Liệu có sự sống nào tồn tại ngoài Trái Đất?” từ lâu đã là một trong những 
vấn đề lớn nhất của thiên văn học và khoa học vũ trụ. Để tiếp cận câu hỏi này, các 
nhà khoa học đã triển khai nhiều hướng nghiên cứu, từ việc tìm kiếm tín hiệu vô 
tuyến, quan sát hành tinh ngoài Hệ Mặt Trời, cho đến phân tích thành phần hóa học 
của các thiên thể. Trong số đó, phương pháp phân tích quang phổ khí quyển hành tinh 
nổi lên như một hướng nghiên cứu đầy hứa hẹn, vì có thể cung cấp thông tin trực tiếp về điều 
kiện vật lý – hóa học liên quan đến khả năng tồn tại sự sống.

Tuy nhiên, phân tích quang phổ của các hành tinh ngoài Hệ Mặt Trời vẫn còn nhiều thách 
thức, đặc biệt do tín hiệu thu được thường rất yếu và bị nhiễu mạnh từ sao chủ cũng như 
các yếu tố quan sát. Hiện chưa có một phương pháp tối ưu nào được công nhận rộng rãi, 
đòi hỏi cộng đồng khoa học phải tìm kiếm các giải pháp mới thông qua cả nghiên cứu lý 
thuyết lẫn thực nghiệm mô phỏng.

Một trong những cách thúc đẩy tiến bộ trong lĩnh vực này là tổ chức các cuộc thi mô 
phỏng dữ liệu quan sát, tạo điều kiện cho các nhà nghiên cứu và kỹ sư dữ liệu thử 
nghiệm, so sánh và cải thiện phương pháp. Cuộc thi Ariel Data Challenge – NeurIPS 2025 
(ADC2025) là một ví dụ tiêu biểu, với mục tiêu xây dựng mô hình dự đoán quang phổ của 
các hành tinh dựa trên dữ liệu quan sát giả lập sát thực tế.

Trong báo cáo này, chúng tôi trình bày các phương pháp đã áp dụng khi tham gia ADC2025, 
bao gồm mô hình hóa bài toán, xây dựng khung quy trình tiền xử lý dữ liệu, tiếp cận
bài toán theo hai hướng: không sử dụng học máy (non-ML) và học máy (ML).

\section{Background}

\subsection{Phương pháp phân tích quang phổ}

Phân tích quang phổ tách ánh sáng thành các bước sóng để xác định thành phần hóa học, nhiệt độ, áp suất và đặc điểm khí quyển của thiên thể.
Các phương pháp chính gồm:

\begin{itemize}
    \item \textbf{Hấp thụ} -- đo suy giảm cường độ tại các bước sóng đặc trưng.
    \item \textbf{Phát xạ} -- ghi nhận bức xạ do vật thể phát ra.
    \item \textbf{Truyền qua} -- quan sát ánh sáng sao đi qua khí quyển hành tinh khi quá cảnh.
    \item \textbf{Phản xạ} -- phân tích ánh sáng phản xạ để xác định suất phản chiếu và bề mặt.
\end{itemize}

Tuy nhiên, để từ những dữ liệu nhiễu muốn có được thông tin chính xác của hành tinh, ta vẫn cần giải pháp trong tiền xử lý và 
phân tích dữ liệu. Trong bối cảnh đó, ADC2025 xuất hiện với mong muốn cùng cộng đồng chung tay thúc đẩy tiến trình nghiên cứu này.


\input{section/background/adc.tex}

\subsection{Tập dữ liệu}

Dữ liệu sử dụng trong nghiên cứu này được cung cấp bởi CAFA 6 gồm các tệp tin chính đóng vai trò
đầu vào và nhãn mục tiêu cho quá trình huấn luyện:
\begin{itemize}
  \item \texttt{train\_sequences.fasta} (Dữ liệu thô):
  \begin{itemize}
    \item Trình tự axit amin (primary sequences) của khoảng 140.000 protein.
    \item Định dạng: FASTA. Mỗi mục gồm mã định danh (EntryID) và chuỗi ký tự đại diện cho các axit amin. (Ví dụ: 
    M, K, T, L,\dots). Đây là dữ liệu đầu chính cho mô hình.
  \end{itemize}
  \item \texttt{train\_terms.tsv} (Dữ liệu nhãn):
  \begin{itemize}
    \item Chứa thông tin gán nhãn chức năng cho các protein trong tập huấn luyện.
    \item Cấu trúc: EntryID <-> GO Terms (Mã chức năng) <-> Aspect (Nhóm chức năng: BPO, MFO, CCO).
    \item Đặc điểm: Một protein có thể tương ứng với nhiều nhãn GO khác nhau (Multi-lable).
  \end{itemize}
  \item \texttt{go-basic.obo} (Cấu trúc Ontology):
  \begin{itemize}
    \item File định nghĩa cấu trúc đồ thị của Gene Ontology, mô tả mối quan hệ cha-con giữa 
    các thuật ngữ chức năng.
  \end{itemize}
  \item \texttt{train\_taxonomy.tsv} (Thông tin loài):
  \begin{itemize}
    \item Cung cấp mã định danh loài (Taxon ID) cho từng protein. Dữ liệu này giúp mô hình phân 
    biệt đặc điểm sinh học giữa các loài khác nhau (ví dụ: vi khuẩn vs. động vật có vú).
  \end{itemize}
\end{itemize}


\section{Phương pháp}

\subsection{Tiền xử lý dữ liệu}
Với số hành tinh $n$, trong thời gian quan sát $t$, có $s$ điểm chụp trong không gian và $wl$ bước sóng, một đầu vào của quy trình tiền xử lý sẽ 
là một tensor có kích thước \textbf{$n \times t \times s \times wl$}. Qua các bước xử lý bên dưới, giả sử ta lấy $a$ bin theo trục thời gian,
lấy trung bình theo trục không gian và $b$ đến $c$ theo trục bước sóng, ta thu được tensor đầu ra có kích thước \textbf{$n \times \frac{t}{a} \times (c-b+1)$}.
% Cụ thể quy trình xử lý với cho từng phương pháp xem ở Phụ lục ~\ref{sec:preprocessing}.


Dưới đây là quy trình tuần tự gồm các bước hiệu chỉnh để xử lý dữ liệu thô chứa
nhiều tạp nhiễu vật lý sử dụng 5 tệp dữ liệu hiệu chuẩn:
\begin{itemize}
  \item dark: ảnh tối, chụp khi không có ánh sáng (đóng shutter), dùng để loại bỏ dòng tối, mỗi giá trị pixel là nhiễu được thêm vào tín hiệu trong 1 giây
  \item dead: xác định dead pixels, đồng thời cũng được dùng để xác định những hot pixels
  \item flat: mỗi giá trị pixel thể hiện độ nhạy của điểm ảnh đó trong cảm biến, dùng để chuẩn hoá sự khác biệt độ nhạy
        giữa các pixel
  \item linear\_corr: đa thức hiệu chỉnh phi tuyến tính của cảm biến
  \item read: ảnh mô tả nhiễu đọc của cảm biến
\end{itemize}
\subsubsection{Đảo ngược chuyển đổi ADC (Analog-to-Digital Converter)}
Tín hiệu thu dưới dạng số nguyên (unit16) nhờ bộ chuyển đổi ADC. Ta phục hồi giá trị thực
(float) của signal bằng công thức:
\begin{equation}
  signal = \frac{raw}{gain} + offset
\end{equation}

Trong đó, $raw$ là giá trị thu được từ các cảm biến, $gain$ là hệ số khuếch đại và $offset$ là độ lệch.

Ví dụ, gain = 0.4369 và offset = -1000 cố định cho ADC2025 (Bạn có thể tìm thấy các giá trị này trong tệp \texttt{adc\_info.csv}).

\subsubsection{Điểm chết (dead pixels) và điểm nóng (hot pixels)}
Trong quá trình xử lý tín hiệu từ cảm biến, hai loại lỗi phổ biến cần được loại bỏ là \textit{điểm chết} (dead pixels) và \textit{điểm nóng} (hot pixels).

\begin{itemize}
  \item \textbf{Điểm chết (dead pixels)} là các pixel không phản ứng với ánh sáng, tức luôn ghi nhận giá trị rất thấp hoặc bằng 0, bất kể tín hiệu thực tế.
  \item \textbf{Điểm nóng (hot pixels)} là các pixel có giá trị bất thường cao, ngay cả trong điều kiện không có ánh sáng, thường do lỗi điện tử hoặc nhiễu nhiệt.
\end{itemize}

Để phát hiện các điểm nóng, thuật toán \texttt{sigma\_clip} từ thư viện \texttt{astropy} được áp dụng trên dữ liệu ảnh tối (\texttt{dark}).
Thuật toán này lặp lại việc loại bỏ các pixel có giá trị vượt quá ngưỡng $5\sigma$ so với trung bình của các pixel lân cận. Sau một số lần lặp, mặt nạ (mask) của các điểm nóng được tạo ra.

Các điểm nóng và điểm chết sau đó được đưa vào mặt nạ hiệu chuẩn \texttt{flat} (ta sẽ nói về \texttt{flat} ở phần cuối cùng), gán giá trị \texttt{NaN} nhằm loại trừ khỏi quá trình hiệu chỉnh:

\subsubsection{Non-linearity correction}
Các pixel trong cảm biến không luôn phản hồi tuyến tính với cường độ ánh sáng chiếu vào.
Khi tín hiệu gần đạt đến mức bão hoà, các pixel sẽ tích điện chậm lại, dẫn đến sai lệch giữa tín hiệu đo được và giá trị thực.
Để hiệu chỉnh sai lệch này, một đa thức (thường là bậc hai hoặc cao hơn) được sử dụng với các hệ số được lưu trong biến \textit{linear\_corr}.
Trong notebook của chúng tôi, chúng tôi áp dụng một đa thức bậc 5 để hiệu chỉnh tín hiệu, như sau:
{\footnotesize
\begin{equation}
  linear\_corr = a \cdot raw^5 + b \cdot raw^4 + c \cdot raw^3 + d \cdot raw^2 + e \cdot raw + f
\end{equation}
}

Trong đó, các hệ số được xác định thông qua dữ liệu trong file \textit{linear\_corr.parquet}.

Chú ý rằng, trước khi áp dụng hiệu chỉnh, tín hiệu nên được clip về 0 vì dữ liệu sẽ có thể có một số pixel với giá trị âm do bước ADC convert ở phần 1 với offset là giá trị âm,
chúng được gây ra bởi nhiễu ngẫu nhiên mà đã được đưa vào tập dữ liệu trong quá trình mô phỏng. Đôi khi điều này xảy ra khi tín hiệu quá yếu. Mặt khác, đa thức mà dữ liệu bài toán cho để hiệu chỉnh
ở trên không hoạt động hiệu quả để xử lý các giá trị âm vì chúng dường như đã được fit từ những giá trị dương. Do đó, nếu giữ nguyên các giá trị âm đó thì
sẽ gây sai lệch lớn trong quá trình hiệu chỉnh.

\subsubsection{Dark frames}
Khung tối là ảnh phơi sáng được chụp với màn trập đóng, ghi lại nhiễu nhiệt và độ lệch của cảm biến. Chúng được sử dụng để loại bỏ dòng tối khỏi ảnh.

\begin{equation}
  signal = signal - dark \cdot \Delta t
\end{equation}

với $\Delta t$ là tổng thời gian phơi sáng của ảnh (thông tin về integration time có thể được tìm thấy trong file \texttt{axis\_info.parquet}).

Khi bạn mở file axis\_info.parquet, bạn sẽ thấy cột \textit{integration\_time} chứa thời gian phơi sáng của từng khung hình chỉ bao gồm các giá trị 0.1s và 4.5s.
Điều này phù hợp với các thông số kỹ thuật của cảm biến FGS1 và AIRS-CH0. Hãy để chúng tôi nói rõ hơn về cách hoạt động của 2 cảm biến này.
Cảm biến hoạt động ở chế độ \textbf{CDS (Correlated Double Sampling)}, nghĩa là sẽ hoạt động theo các chu kỳ gọi là \texttt{integration ramp}.
Mỗi integration ramp bắt đầu với detector ở trạng thái ban đầu, tích điện được kích hoạt thời điểm này. Sau 0.1 giây, nó sẽ đọc tín hiệu đầu tiên (NDR0). Quá trình đọc 1 frame mất 0.1s, nhưng giả thuyết bài toán này không sử dụng phương pháp đọc tuần tự
mà thay vào đó, tất cả pixel được đọc đồng thời. Sau khi NDR0 được chụp, detector sẽ chuyển sang giai đoạn đợi chờ để thu thập ánh sáng. Khoảng thời gian chờ này chính là cột integration time trong tập dữ liệu.
Sau pha chờ này, detector sẽ đọc tín hiệu lần thứ hai (NDR1) và cuối cùng là reset detector để chuẩn bị cho ramp tiếp theo. Quá trình reset này cũng cần 0.1 giây.
Bạn có thể kiểm tra điều này bằng cách so sánh dòng thời gian của các khung hình với thời gian tích hợp (trong file \texttt{axis\_info.parquet}). Để tham khảo, đây là dòng thời gian của hai máy dò:
\begin{itemize}
  \item \textbf{FGS1 (integration time = wait = \SI{0.1}{\second}):}

        \begin{tabbing}
          \hspace{4em} \= \kill
          \texttt{|--ground--|--NDR0--|--wait--|--NDR1--|--reset--|} \\
          \texttt{0.0} \> \texttt{0.1 \hspace{2em} 0.2 \hspace{1em} 0.3 \hspace{1em} 0.4 \hspace{2em} 0.5}
        \end{tabbing}

  \item \textbf{AIRS (integration time = wait = \SI{4.5}{\second}):}

        \begin{tabbing}
          \hspace{4em} \= \kill
          \texttt{|--ground--|--NDR0--|--wait--|--NDR1--|--reset--|} \\
          \texttt{0.0} \> \texttt{0.1 \hspace{2em} 0.2  \hspace{1em} 4.7 \hspace{1em} 4.8 \hspace{2em} 4.9}
        \end{tabbing}
\end{itemize}

Như vậy, ta cần chú ý ở đây rằng: thời gian phơi sáng các frame đầu tiên ở 2 cảm biến sẽ đều là 0.1 giây (do pha ground kéo dài 0.1 giây). Sự khác nhau giữa 2 cảm biến nằm ở frame2.
Với giả thuyết các pixel được chụp đồng thời thì tổng thời gian phơi sáng $\Delta t$ của frame 2 ở FGS1 là 0.2 giây, với AIRS là 4.6 giây.

\subsubsection{Flat field correction}
Lý do cho việc hiệu chỉnh flat field là để loại bỏ sự khác biệt về độ nhạy giữa các pixel trong cảm biến.
Cảm biến trong máy ảnh không hoàn hảo, có pixel nhạy hơn, có pixel kém nhạy hơn, có thể có bóng mờ, bụi, hay sự không đồng đều trong hệ quang học.
Do đó, khi chụp một vùng sáng đồng đều (ví dụ: ánh sáng trắng trải đều), ảnh thu được không hề đồng đều, mà có vùng sáng – vùng tối do sự khác biệt trong cảm biến.
Vì vậy, để thu được độ nhạy cảm mỗi pixel, người ta chụp một ảnh với nguồn sáng đồng đều, ảnh thu được gọi là flat field, mỗi  pixel trong ảnh này thể hiện độ nhạy sáng của pixel đó.
Tín hiệu mà ta thu được sẽ tỉ lệ với độ nhạy của pixel nhân với tín hiệu thực tế từ bản chất ánh sáng đó.
Do đó, với một ảnh bất kỳ thu được, để hiệu chỉnh nó để thu được tín hiệu thực từ nguồn sáng, ta phải thực hiện chia cho flat field (độ nhạy sáng của mỗi pixel).

Tín hiệu pixel khác nhau do hiệu suất quang học khác nhau. Cần hiệu chỉnh bằng cách
dùng flat mapping để chuẩn hoá:
\begin{equation}
  signal_{corrected} = \frac{signal}{flat\_field}
\end{equation}

\begin{figure}[htbp]
  \centering
  \includegraphics[width=0.8\linewidth]{figures/preprocess_pipeline.png}
  \caption{Pipeline tiền xử lý dữ liệu}
  \label{fig:preprocess_pipeline}
\end{figure}
\subsubsection{Correlated Double Sampling (CDS)}
Trước khi tính CDS, ta cần làm gọn dữ liệu bằng cách loại bỏ đi các phần không quan trọng của frame, cũng như làm một số thao tác để lấy ra con số biểu diễn chung nhất cho một frame.
Sau đó, ta tính CDS bằng cách lấy hiệu ảnh chụp được lúc kết thúc và bắt đầu của mỗi ramp, cụ thể là:


\begin{equation}
  CDS = signal_{end} - signal_{start}
\end{equation}

\subsubsection{Binning}
Binning là quá trình kết hợp tín hiệu của nhiều bước thời gian thành một để tiết kiệm không gian bộ nhớ cũng như tính toán. Điều này không làm giảm hiệu quả của model vì ta nhận thấy,
các hành tinh di chuyển rất chậm, do đó, các ảnh liên tiếp nhau không có sự khác nhau nhiều nên việc binning sẽ không làm mất đi thông tin quan trọng mà còn giảm được nhiễu nhờ cách tính trung bình trên mỗi bin.


\subsection{Tiếp cận theo hướng không dùng học máy}
\subsubsection{Naive Prior Baseline: Frequency-based GO Term Prediction} \mbox{}
\label{sec:baseline_naive}
Để thiết lập một mốc tham chiếu (baseline) đơn giản, chúng tôi xây dựng 
một phương pháp dự đoán dựa hoàn toàn trên thống kê tần suất nhãn trong tập 
huấn luyện, không sử dụng bất kỳ đặc trưng nào từ chuỗi protein hay embedding. 
Ý tưởng cốt lõi là khai thác \textit{prior distribution} của các nhãn 
Gene Ontology (GO): các GO term xuất hiện càng thường xuyên trong dữ liệu huấn 
luyện được giả định có xác suất cao hơn để xuất hiện ở các protein chưa biết nhãn.

\paragraph{Ước lượng phân phối nhãn theo từng phân hệ (aspect).}
Gọi $a \in \{P, C, F\}$ lần lượt tương ứng với \textit{Biological Process}, \textit{Cellular Component} và \textit{Molecular Function}. Từ tập huấn luyện, ta đếm số lần xuất hiện của mỗi GO term $t$ trong aspect $a$:
\begin{equation}
c_a(t) = \#\{(\cdot, t)\mid \text{aspect}=a\}, 
\end{equation}
và tổng số annotation trong aspect:
\begin{equation}
N_a = \sum_{t} c_a(t).
\end{equation}
Xác suất tiên nghiệm của term $t$ trong aspect $a$ được ước lượng bằng tần suất chuẩn hoá:
\begin{equation}
p_a(t)=\frac{c_a(t)}{N_a}.
\end{equation}

\paragraph{Chọn top-$k$ GO term phổ biến nhất.}
Với mỗi aspect $a$, ta chọn tập $\mathcal{T}^{(k)}_a$ gồm $k$ term có $c_a(t)$ lớn nhất:
\begin{equation}
\mathcal{T}^{(k)}_a = \text{TopK}_{t}\big(c_a(t)\big).
\end{equation}

\paragraph{Inference}
Với mỗi protein trong tập test $x$, baseline này không phân biệt giữa các protein mà xuất ra cùng một tập dự đoán cho mọi $x$. Cụ thể, với mỗi aspect $a$, ta dự đoán tất cả các term trong $\mathcal{T}^{(k)}_a$ với điểm tin cậy bằng xác suất tiên nghiệm $p_a(t)$:
\begin{equation}
\hat{s}(x,t)=
\begin{cases}
p_a(t), & \text{nếu }\, t \in \mathcal{T}^{(k)}_a,\\
\text{không xuất ra}, & \text{ngược lại}.
\end{cases}
\end{equation}

\paragraph{Vai trò và hạn chế.}
Ưu điểm của baseline này là đơn giản, dễ tái lập và gần như không tốn chi phí tính toán. Tuy nhiên, do dự đoán không phụ thuộc vào đặc trưng protein, phương pháp không thể cá nhân hoá dự đoán theo chuỗi và chỉ phù hợp như một mốc tham chiếu để so sánh với các phương pháp học có điều kiện theo protein.

\subsubsection{Phương pháp tiếp cận dựa trên sự tương đồng (Homology-based Approach)}\mbox{}

Chúng tôi sử dụng phương pháp tin sinh học kinh điển: Gióng hàng trình tự (Sequence Alignment). Nguyên lý cốt lõi của phương pháp này dựa trên giả thuyết sinh học \textit{``Guilt by Association''} (Tương đồng cấu trúc dẫn đến tương đồng chức năng). 

Cụ thể, nếu protein $A$ (trong tập Test) có trình tự axit amin tương đồng cao với protein $B$ (đã biết chức năng trong tập Train), khả năng cao $A$ sẽ thừa hưởng các chú giải chức năng của $B$. Do đó, bài toán dự đoán chức năng được chuyển về bài toán tìm kiếm láng giềng gần nhất trong không gian trình tự sinh học.

\paragraph{Thuật toán BLAST (Basic Local Alignment Search Tool)}
BLAST là tiêu chuẩn vàng trong tìm kiếm tương đồng trình tự cục bộ. Thay vì sử dụng quy hoạch động toàn cục (như Needleman-Wunsch) tốn kém chi phí tính toán $O(m \times n)$, BLAST sử dụng phương pháp heuristic dựa trên thống kê để tìm các vùng tương đồng cục bộ (Local Alignment) có ý nghĩa nhất.

Quy trình hoạt động và cơ sở toán học của BLAST bao gồm 3 thành phần chính:

\texttt{1. Ma trận chấm điểm (Scoring System):}
Để định lượng mức độ giống nhau giữa hai axit amin $i$ và $j$, BLAST sử dụng ma trận điểm thay thế (thường là BLOSUM62). Điểm số $S_{ij}$ được tính dựa trên tỷ lệ \textit{log-odds}:

\begin{equation}
S_{ij} = \frac{1}{\lambda} \log \left( \frac{p_{ij}}{q_i q_j} \right)
\end{equation}

Trong đó:
\begin{itemize}
    \item $p_{ij}$: Xác suất quan sát thấy axit amin $i$ và $j$ thay thế cho nhau trong các chuỗi tương đồng thực sự (do bảo tồn tiến hóa).
    \item $q_i, q_j$: Tần suất xuất hiện ngẫu nhiên (nền) của axit amin $i$ và $j$.
    \item $\lambda$: Hệ số tỷ lệ.
\end{itemize}
Nếu $S_{ij} > 0$, sự thay thế được coi là tương đồng; ngược lại là không tương đồng.

\texttt{2. Cơ chế Seed-and-Extend:}
Thuật toán hoạt động qua hai bước:
\begin{itemize}
    \item \textbf{Gieo hạt (Seeding):} Chia chuỗi truy vấn thành các đoạn ngắn (k-mers, thường $k=3$ cho protein). Tìm các vị trí trong cơ sở dữ liệu khớp với các k-mers này (hoặc có điểm số thay thế vượt ngưỡng $T$).
    \item \textbf{Mở rộng (Extension):} Từ các hạt giống, thuật toán mở rộng sang hai phía để hình thành Cặp đoạn điểm cao (HSP --- High-scoring Segment Pair). Điểm số tích lũy $S$ của đoạn liên kết được tính bằng tổng điểm các cặp axit amin trừ đi điểm phạt khoảng trống (gap penalty):
    \begin{equation}
    S = \sum_{k} S(x_k, y_k) - (G_{open} + L_{gap} \cdot G_{ext})
    \end{equation}
    Quá trình mở rộng dừng lại khi điểm số suy giảm quá mức $X$ so với giá trị đỉnh (drop-off heuristic).
\end{itemize}

\texttt{3. Đánh giá thống kê (E-value):}
Để loại bỏ các kết quả ngẫu nhiên, BLAST sử dụng chỉ số E-value (Expectation value). E-value biểu thị số lượng các so khớp có điểm số $\ge S$ mà ta kỳ vọng tìm thấy \textit{ngẫu nhiên} trong một cơ sở dữ liệu kích thước $N$:

\begin{equation}
E = K \cdot m \cdot n \cdot e^{-\lambda S}
\end{equation}

Trong đó $m, n$ là độ dài chuỗi truy vấn và cơ sở dữ liệu; $K, \lambda$ là hằng số thống kê Gumbel. Một kết quả được coi là có ý nghĩa sinh học nếu $E$ rất nhỏ (ví dụ $E < 10^{-5}$).
\\
\\
Hạn chế đối với CAFA 6 đó là mặc dù có độ chính xác cao, độ phức tạp tính toán của BLAST trở thành rào cản lớn đối với dữ liệu Big Data. Với tập dữ liệu hàng trăm nghìn chuỗi của CAFA 6, việc thực hiện so khớp tất cả với tất cả (All-vs-All) bằng BLAST có thể mất hàng tuần xử lý trên CPU thông thường, không khả thi trong giới hạn thời gian của cuộc thi.

\paragraph{Sử dụng DIAMOND làm giải pháp thay thế}

Để khắc phục hạn chế về hiệu năng tính toán của BLAST trên dữ liệu lớn, chúng tôi sử dụng \textbf{DIAMOND} (Double Index Alignment of Next-generation Sequencing Data). Đây là thuật toán được tối ưu hóa đặc biệt cho dữ liệu giải trình tự thế hệ mới, cho phép tốc độ xử lý nhanh hơn từ \textbf{500 đến 20.000 lần} so với BLAST trong khi vẫn duy trì độ nhạy tương đương ở các thiết lập tiêu chuẩn.

Cơ sở toán học và thuật toán của DIAMOND dựa trên hai cải tiến cốt lõi so với mô hình \textit{Seed-and-Extend} truyền thống:

\texttt{1. Bảng chữ cái rút gọn (Reduced Alphabet):}
Trong khi BLAST so sánh chính xác trên không gian 20 axit amin chuẩn $\Sigma_{std}$, DIAMOND thực hiện tìm kiếm hạt giống trên một không gian rút gọn $\Sigma_{red}$ (thường gồm 11 nhóm).
Ánh xạ $\Phi$ được định nghĩa dựa trên tính chất hóa lý của axit amin:

\begin{equation}
\Phi: \Sigma_{std} \to \Sigma_{red}
\end{equation}

Ví dụ: Các axit amin kỵ nước (Hydrophobic) hoặc tích điện dương (Positively Charged) được gom nhóm:
\[ \Phi(\{L, V, I, M\}) \to \text{Hydrophobic}, \quad \Phi(\{K, R\}) \to \text{Positive} \]

Điều kiện khớp hạt giống (seed match) giữa đoạn chuỗi $S_1$ và $S_2$ tại vị trí $k$ trở thành:
\begin{equation}
\text{Match}(S_1[k], S_2[k]) \iff \Phi(S_1[k]) = \Phi(S_2[k])
\end{equation}

Việc này làm tăng xác suất tìm thấy các hạt giống (seeds) ngay cả khi có đột biến thay thế axit amin, cho phép sử dụng các hạt giống dài hơn và có khoảng cách (spaced seeds) để tăng tốc độ lọc mà không làm giảm độ nhạy.

\texttt{2. Đánh chỉ mục kép (Double Indexing):}
Khác với BLAST chỉ đánh chỉ mục cơ sở dữ liệu (Database), DIAMOND đánh chỉ mục cho cả Chuỗi truy vấn (Query) và Cơ sở dữ liệu cùng lúc.
Thuật toán sắp xếp các hạt giống từ cả hai nguồn theo thứ tự từ điển. Quá trình tìm kiếm trở thành việc duyệt tuyến tính qua hai danh sách đã sắp xếp, giúp tối ưu hóa việc truy cập bộ nhớ cache (cache locality) và giảm thiểu chi phí truy cập ngẫu nhiên.

\texttt{3. Chuyển giao điểm số (Score Propagation):}
Sau khi tìm được các protein tương đồng từ tập huấn luyện, điểm số dự đoán cho protein mục tiêu được tính:

\begin{equation}
\resizebox{0.9\columnwidth}{!}{$\displaystyle
Score(P_{test}, F) = \max_{P_{train} \in \text{Hits}} \left( \frac{\text{pident}(P_{test}, P_{train})}{100} \times y_{train, F} \right)
$}
\end{equation}

Trong đó $\text{pident}$ là tỷ lệ phần trăm axit amin giống hệt, $y_{train, F} \in \{0, 1\}$ là nhãn của protein huấn luyện.
Công thức này đảm bảo rằng các protein càng giống nhau về cấu trúc thì độ tin cậy của việc chuyển giao chức năng càng cao.

\subsection{Tiếp cận theo hướng sử dụng học máy}
Chúng tôi đã thử một số phương pháp học máy khác nhau, gồm: K-Nearest Neighbors (KNN), Gaussian Process Regressor, phương pháp denoise bằng Auto Encoder. Tuy nhiên, kết quả thu được không khả quan, do dữ liệu có quá nhiều nhiễu và không đủ thông tin để mô hình hóa chính xác.


\section{Thực nghiệm}

\subsection{Tối ưu các hệ số mô hình non-ML}

Ý tưởng của phương pháp non-ML đơn giản nhưng để thực nghiệm
hiệu quả thì cần phải chọn các tham số phù hợp dựa vào các quan sát trên dữ liệu.

Như đã đề cập ở trên, chúng tôi nhân đoạn tín hiệu transit với $1 + s$ và kỳ vọng
đoạn tín hiệu thu được sẽ mượt như khi không có hành tinh nào đi qua sao chủ. Tín hiệu
này sẽ được xấp xỉ bằng một đa thức. Việc chọn bậc của đa thức này ảnh hưởng lớn đến
độ chính xác của mô hình. Nhìn vào dữ liệu như hình $\ref{fig:poly_degree}$, ta thấy rằng
\begin{figure}[htbp]
    \centering
    \includegraphics[width=0.8\linewidth]{figures/poly_degree.png}
    \caption{Hình dạng phổ biến của flux theo thời gian trong tập dữ liệu}
    \label{fig:poly_degree}
\end{figure}
sau khi nhân phần lõm ở giữa với $1 + s$ thì tín hiệu không phải lúc nào cũng là 1 đường tuyến tính.
Hình dạng phổ biến là bậc 1, 2, 3.
Chúng tôi thực nghiệm với các giá trị bậc đa thức khác nhau cho kết quả như bảng $\ref{tab:poly_degree_score}$.

\begin{table}[htbp]
    \centering
    \begin{tabular}{|c|c|c|c|c|c|c|}
        \hline
        Bậc đa thức & 1 & 2 & 3 & 4 & 5 & 10 \\
        \hline
        Điểm & 0.314 & 0.315 & 0.322 & 0.310 & 0.304 & 0.115 \\
        \hline
    \end{tabular}
    % \caption{Ảnh hưởng của bậc đa thức đến điểm mô hình non-ML}
    \label{tab:poly_degree_score}
\end{table}

Sau khi fit tín hiệu với một đa thức bậc 3, ta tìm $s$ tối ưu (hiệu trị tuyệt đối nhỏ nhất),
chúng tôi dự đoán transit depth dựa vào $s$.

\begin{figure}[htbp]
    \centering
    \includegraphics[width=0.8\linewidth]{figures/explain_SCALE.png}
    \caption{Mối quan hệ giữa s, mục tiêu cần dự đoán}
    \label{fig:explain_scale}
\end{figure}

Nhận thấy rằng, nếu lấy giá trị dự đoán theo $\frac{s}{1 + s}$ 
cho ra kết quả thấp hơn so với lấy $s * SCALE$. Hơn nữa, khi ta tính giá trị $SCALE = truth.mean / s$
với mỗi điểm dữ liệu, thì giá trị này sẽ gần bằng nhau cho tất cả các hành tinh.
Chạy thực nghiệm cho thấy, sử dụng $SCALE = 0.9396$ cho điểm cao nhất.

Dựa vào công thức tính điểm, để tối ưu điểm không chỉ cần dự đoán chính xác mà còn cần
dự đoán độ không chắc chắn phù hợp. Giả định rằng, giá trị dự đoán của mô hình khớp với
trung bình của quang phổ cần dự đoán, như vậy
\begin{equation}
    \sigma^* = argmax_{\sigma}\sum_{i = 1}^{283} \left( log(\sigma^2) + \frac{(truth[i] - \mu_{truth})^2}{\sigma^2} \right)
\end{equation}

Dễ thấy, hàm mục tiêu là tổng 2 hàm lồi, 
\begin{equation}
    (\sigma^*)^2 = \sum_{i = 1}^{283} (truth[i] - \mu_{truth})^2 = (\sigma_{truth})^2
\end{equation}

Nhận thấy rằng, giá trị $\sigma$ tối ưu cho từng hành tinh là khác nhau và có liên quan 
đến tín hiệu theo trục thời gian. Chúng tôi sử dụng mô hình phụ LinearRegression với đầu vào là tín hiệu,
đầu ra là $\sigma_{truth}$, và thực hiện huấn luyện mô hình này trên tập dữ liệu huấn luyện.
Kết quả tăng 0.004 so với việc sử dụng giá trị $\sigma$ cố định cho tất cả các hành tinh.

\subsection{Kết quả}


\begin{figure}[h]
    \centering
    \includegraphics[width=0.9\linewidth]{figures/regressor.png}
    \caption{Kết quả của mô hình Regressor so với non ML}
    \label{fig:regressor}
\end{figure}

\begin{figure}[h]
  \centering
  \includegraphics[width=0.45\textwidth]{figures/raw_dips.png}
  \caption{Các bước sóng chưa qua làm mịn}
  \label{fig:raw_dips}
\end{figure}

\begin{figure}[h]
  \centering
  \includegraphics[width=0.5\textwidth]{figures/gp_ae_dip.png}
  \caption{Các bước sóng đã qua làm mịn}
  \label{fig:smoothed_dips}
\end{figure}

Phương pháp non-ML dự đoán quang phổ và LinearRegression dự đoán độ không chắc chắn
 cho điểm trên tập valid là 0.75, điểm trên tập test là 0.326. Đạt hạng 56/460 đội (tính đến ngày 9/8/2025).

Với các phương pháp ML cho dự đoán quang phổ,

Chúng tôi thử với mô hình Regressor, lấy thuộc tính là giá trị flux và một số thuộc tính
của gradient của flux có liên quan mạnh đến độ sâu transit như hình $\ref{fig:regressor}$


\begin{figure}[h]
  \centering
  \includegraphics[width=0.4\textwidth]{figures/non_ml_dip.png}
  \caption{Các bước sóng dự đoán của mô hình non-ML}
  \label{fig:non_ml_dip}
\end{figure}


với điểm trên tập valid là 0.734, điểm trên tập test là 0.289, nguyên nhân có thể là do sự khác biệt trong phân phối dữ liệu giữa các tập
và chọn thuộc tính chưa phù hợp.



Sự khác biệt về phân phối dữ liệu càng thể hiện rõ hơn khi chúng tôi thử với KNN, với khoảng cách trung bình các điểm dữ liệu trong tập train (với k = 2) là 50.
Chúng tôi kết hợp KNN và phương pháp non-ML, với điểm dữ liệu trong tập test có khoảng cách với các điểm trong trong KNN dưới 40, chúng tôi nội suy từ tập train, những điểm lớn hơn
thì chúng tôi sử dụng phương pháp non-ML. Kết quả test là 0.322, có hơn phương pháp non-ML nhưng độ chênh lệch quá nhỏ.



Chúng tôi cũng thử làm mịn kết quả dự đoán của 283 bước sóng bằng Gaussian Process (GP) và AutoEncoder (AE).

\begin{figure}[h]
  \centering
  \includegraphics[width=0.5\textwidth]{figures/pca.png}
  \caption{Các bước sóng đã qua làm mượt bằng PCA}
  \label{fig:pca_smoothed_dips}
\end{figure}

Như trong hình ~\ref{fig:raw_dips}, các bước sóng chưa làm mịn có nhiều nhiễu và không đồng nhất. Để cải thiện, chúng tôi đã áp dụng 
Gaussian Process (GP) và AutoEncoder (AE) để làm mịn, khử nhiễu các bước sóng này. Kết quả là các bước sóng trở nên đồng nhất hơn, 
như trong hình ~\ref{fig:smoothed_dips}.

\begin{figure}[h]
  \centering
  \includegraphics[width=0.5\textwidth]{figures/ml_catch.png}
  \caption{Dự đoán của mô hình ML bám sát được hình dạng của ground truth}
  \label{fig:ml_catch}
\end{figure}

Đồng thời, GP cũng sinh ra các ước lượng không chắc chắn cho các bước sóng, cho phép chúng tôi đánh giá độ tin cậy $\sigma$ của các dự đoán.
So với bước sóng được dự đoán bằng phương pháp non-ML , như trong hình ~\ref{fig:non_ml_dip}, các bước sóng này đã sát với ground truth hơn,
0.00356 so với 0.0024.


\begin{figure}[h]
  \centering
  \includegraphics[width=0.4\textwidth]{figures/non_ml_catch.png}
  \caption{Dự đoán của mô hình non-ML không bám sát được hình dạng của ground truth}
  \label{fig:non_ml_catch}
\end{figure}

Một ý tưởng khác để làm mịn bước sóng là sử dụng PCA tìm ra những component chính của các bước sóng này.
Kết quả làm mượt bằng PCA cho thấy những tiềm năng của phương pháp này trong việc cải thiện độ chính xác của dự đoán. Bằng việc sử dụng 5 component
chính, chúng tôi đã đạt được những cải thiện đáng kể trong việc giảm thiểu nhiễu và tăng cường độ chính xác cho các bước sóng dự đoán.




Dù việc sử dụng PCA đã mang lại những cải thiện đáng kể trên valid, nhưng vẫn còn nhiều thách thức trong việc xử lý dữ liệu nhiễu và không đồng nhất. PCA với
5 component đem lại kết quả thấp trên test set so với valid set có thể do các tập trong test set phân tán hơn, dẫn đến việc mô hình không thể tổng quát tốt hơn 
cho các bước sóng này.




So với sử dụng mean của toàn bộ quang phổ như non-ML, việc dự đoán từng bước sóng cải thiện việc bắt được cấu trúc của các bước sóng quang phổ phức tạp
, như hình ~\ref{fig:ml_catch}. So với việc dự đoán từng bước sóng, phương pháp non-ML chỉ cho ra được một giá trị trung bình cho toàn bộ quang phổ,
như hình ~\ref{fig:non_ml_catch}. Tuy nhiên, việc dự đoán từng bước sóng cũng làm tăng độ nhiễu và độ không đồng nhất của các bước sóng này, dẫn đến việc khó khăn trong việc
tổng quát khiến những phương pháp chúng tôi thử nghiệm không đạt được kết quả tốt hơn so với non-ML.

\section{Kết luận}

Trong bài báo này, chúng tôi đã trình bày phương pháp non-ML
để dự đoán quang phổ và linear regression để dự đoán độ không chắc chắn.
Phương pháp này sử dụng các kỹ thuật xử lý tín hiệu truyền thống và cho kết 
quả tốt hơn các phương pháp ML (KNN, GP, AE) chúng tôi đã thử.

Tuy vậy, các phương pháp học máy hiện đang dùng cho kết quả tốt hơn trên tập valid. Do một vài
nguyên nhân khiến điểm test chưa tốt. Chúng tôi sẽ tiếp tục nghiên cứu để cải thiện hơn nữa
các phương pháp này trong tương lai.

\input{section/cite.tex}

% \clearpage

\appendix
% \subsection{Chi tiết về tiền xử lý dữ liệu} \label{sec:preprocessing}
% \subsubsection{Phân tích dữ liệu đầu ra}

% Phân tích file \textbf{wavelength.csv} để lấy ra bước sóng có trong test, phân tích bước sóng có trong tập \textbf{axis\_info.parquet}, ta biết được
% bước sóng đầu $wl_1$ trong kết quả đầu ra mong muốn là của FGS1, 282 bước sóng còn lại lần lượt tương ứng với bước sóng từ 321 tới 39 (ứng với $wl_2$ tới $wl_{283}$).

% \subsubsection{Phương pháp không học máy}

% Dữ liệu đầu vào là tensor dạng (n, t, s, wl), trong đó n là số hành tinh, t là thời gian quan sát, s là số điểm chụp trong không gian và wl là bước sóng.
% Ta xử lý dữ liệu từ \textbf{AIRS-CH0} bằng cách áp dụng các bước tiền xử lý, từ file parquet ta reshape về dạng [11250, 32, 356], lấy binning là 30 theo thời gian, 
% lấy trung bình theo không gian và lấy bước sóng từ 39 đến 321, thu được tensor [187, 282]. Với dữ liệu từ \textbf{FGS1}, ta reshape về dạng [11250, 32, 32], lấy binning là 30 theo thời gian, 
% lấy trung bình theo không gian và toàn bộ bước sóng, đầu ra sẽ có dạng [187, 1].

% table, column is members, row is % contribute
\subsection{Tỉ lệ đóng góp}

\begin{table}[h]
    \centering
    \begin{tabular}{|c|c|c|c|c|}
        \hline
        Thành viên & Phan Bá Thọ & Nguyễn Quốc Huy & Trần Tuấn Anh \\
        \hline
        Tỉ lệ đóng góp & 34\% & 33\% & 33\% \\
        \hline
    \end{tabular}
    \caption{Tỉ lệ đóng góp của các thành viên trong nhóm}
\end{table}

\input{section/cite.tex}


\end{document}


