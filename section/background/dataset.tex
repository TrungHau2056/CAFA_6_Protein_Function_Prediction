\subsection{Tập dữ liệu}

Dữ liệu sử dụng trong nghiên cứu này được cung cấp bởi CAFA 6 gồm các tệp tin chính đóng vai trò
đầu vào và nhãn mục tiêu cho quá trình huấn luyện:
\begin{itemize}
  \item \texttt{train\_sequences.fasta} (Dữ liệu thô):
  \begin{itemize}
    \item Trình tự axit amin (primary sequences) của khoảng 140.000 protein.
    \item Định dạng: FASTA. Mỗi mục gồm mã định danh (EntryID) và chuỗi ký tự đại diện cho các axit amin. (Ví dụ: 
    M, K, T, L,\dots). Đây là dữ liệu đầu chính cho mô hình.
  \end{itemize}
  \item \texttt{train\_terms.tsv} (Dữ liệu nhãn):
  \begin{itemize}
    \item Chứa thông tin gán nhãn chức năng cho các protein trong tập huấn luyện.
    \item Cấu trúc: EntryID <-> GO Terms (Mã chức năng) <-> Aspect (Nhóm chức năng: BPO, MFO, CCO).
    \item Đặc điểm: Một protein có thể tương ứng với nhiều nhãn GO khác nhau (Multi-lable).
  \end{itemize}
  \item \texttt{go-basic.obo} (Cấu trúc Ontology):
  \begin{itemize}
    \item File định nghĩa cấu trúc đồ thị của Gene Ontology, mô tả mối quan hệ cha-con giữa 
    các thuật ngữ chức năng.
  \end{itemize}
  \item \texttt{train\_taxonomy.tsv} (Thông tin loài):
  \begin{itemize}
    \item Cung cấp mã định danh loài (Taxon ID) cho từng protein. Dữ liệu này giúp mô hình phân 
    biệt đặc điểm sinh học giữa các loài khác nhau (ví dụ: vi khuẩn vs. động vật có vú).
  \end{itemize}
\end{itemize}
