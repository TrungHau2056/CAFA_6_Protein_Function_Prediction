\subsection{Tối ưu các hệ số mô hình non-ML}

Ý tưởng của phương pháp non-ML đơn giản nhưng để thực nghiệm
hiệu quả thì cần phải chọn các tham số phù hợp dựa vào các quan sát trên dữ liệu.

Như đã đề cập ở trên, chúng tôi nhân đoạn tín hiệu transit với $1 + s$ và kỳ vọng
đoạn tín hiệu thu được sẽ mượt như khi không có hành tinh nào đi qua sao chủ. Tín hiệu
này sẽ được xấp xỉ bằng một đa thức. Việc chọn bậc của đa thức này ảnh hưởng lớn đến
độ chính xác của mô hình. Nhìn vào dữ liệu như hình $\ref{fig:poly_degree}$, ta thấy rằng
\begin{figure}[htbp]
    \centering
    \includegraphics[width=0.8\linewidth]{figures/poly_degree.png}
    \caption{Hình dạng phổ biến của flux theo thời gian trong tập dữ liệu}
    \label{fig:poly_degree}
\end{figure}
sau khi nhân phần lõm ở giữa với $1 + s$ thì tín hiệu không phải lúc nào cũng là 1 đường tuyến tính.
Hình dạng phổ biến là bậc 1, 2, 3.
Chúng tôi thực nghiệm với các giá trị bậc đa thức khác nhau cho kết quả như bảng $\ref{tab:poly_degree_score}$.

\begin{table}[htbp]
    \centering
    \begin{tabular}{|c|c|c|c|c|c|c|}
        \hline
        Bậc đa thức & 1 & 2 & 3 & 4 & 5 & 10 \\
        \hline
        Điểm & 0.314 & 0.315 & 0.322 & 0.310 & 0.304 & 0.115 \\
        \hline
    \end{tabular}
    % \caption{Ảnh hưởng của bậc đa thức đến điểm mô hình non-ML}
    \label{tab:poly_degree_score}
\end{table}

Sau khi fit tín hiệu với một đa thức bậc 3, ta tìm $s$ tối ưu (hiệu trị tuyệt đối nhỏ nhất),
chúng tôi dự đoán transit depth dựa vào $s$.

\begin{figure}[htbp]
    \centering
    \includegraphics[width=0.8\linewidth]{figures/explain_SCALE.png}
    \caption{Mối quan hệ giữa s, mục tiêu cần dự đoán}
    \label{fig:explain_scale}
\end{figure}

Nhận thấy rằng, nếu lấy giá trị dự đoán theo $\frac{s}{1 + s}$ 
cho ra kết quả thấp hơn so với lấy $s * SCALE$. Hơn nữa, khi ta tính giá trị $SCALE = truth.mean / s$
với mỗi điểm dữ liệu, thì giá trị này sẽ gần bằng nhau cho tất cả các hành tinh.
Chạy thực nghiệm cho thấy, sử dụng $SCALE = 0.9396$ cho điểm cao nhất.

Dựa vào công thức tính điểm, để tối ưu điểm không chỉ cần dự đoán chính xác mà còn cần
dự đoán độ không chắc chắn phù hợp. Giả định rằng, giá trị dự đoán của mô hình khớp với
trung bình của quang phổ cần dự đoán, như vậy
\begin{equation}
    \sigma^* = argmax_{\sigma}\sum_{i = 1}^{283} \left( log(\sigma^2) + \frac{(truth[i] - \mu_{truth})^2}{\sigma^2} \right)
\end{equation}

Dễ thấy, hàm mục tiêu là tổng 2 hàm lồi, 
\begin{equation}
    (\sigma^*)^2 = \sum_{i = 1}^{283} (truth[i] - \mu_{truth})^2 = (\sigma_{truth})^2
\end{equation}

Nhận thấy rằng, giá trị $\sigma$ tối ưu cho từng hành tinh là khác nhau và có liên quan 
đến tín hiệu theo trục thời gian. Chúng tôi sử dụng mô hình phụ LinearRegression với đầu vào là tín hiệu,
đầu ra là $\sigma_{truth}$, và thực hiện huấn luyện mô hình này trên tập dữ liệu huấn luyện.
Kết quả tăng 0.004 so với việc sử dụng giá trị $\sigma$ cố định cho tất cả các hành tinh.