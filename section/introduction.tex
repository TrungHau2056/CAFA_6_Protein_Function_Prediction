\section{Giới thiệu}

Câu hỏi “Liệu có sự sống nào tồn tại ngoài Trái Đất?” từ lâu đã là một trong những 
vấn đề lớn nhất của thiên văn học và khoa học vũ trụ. Để tiếp cận câu hỏi này, các 
nhà khoa học đã triển khai nhiều hướng nghiên cứu, từ việc tìm kiếm tín hiệu vô 
tuyến, quan sát hành tinh ngoài Hệ Mặt Trời, cho đến phân tích thành phần hóa học 
của các thiên thể. Trong số đó, phương pháp phân tích quang phổ khí quyển hành tinh 
nổi lên như một hướng nghiên cứu đầy hứa hẹn, vì có thể cung cấp thông tin trực tiếp về điều 
kiện vật lý – hóa học liên quan đến khả năng tồn tại sự sống.

Tuy nhiên, phân tích quang phổ của các hành tinh ngoài Hệ Mặt Trời vẫn còn nhiều thách 
thức, đặc biệt do tín hiệu thu được thường rất yếu và bị nhiễu mạnh từ sao chủ cũng như 
các yếu tố quan sát. Hiện chưa có một phương pháp tối ưu nào được công nhận rộng rãi, 
đòi hỏi cộng đồng khoa học phải tìm kiếm các giải pháp mới thông qua cả nghiên cứu lý 
thuyết lẫn thực nghiệm mô phỏng.

Một trong những cách thúc đẩy tiến bộ trong lĩnh vực này là tổ chức các cuộc thi mô 
phỏng dữ liệu quan sát, tạo điều kiện cho các nhà nghiên cứu và kỹ sư dữ liệu thử 
nghiệm, so sánh và cải thiện phương pháp. Cuộc thi Ariel Data Challenge – NeurIPS 2025 
(ADC2025) là một ví dụ tiêu biểu, với mục tiêu xây dựng mô hình dự đoán quang phổ của 
các hành tinh dựa trên dữ liệu quan sát giả lập sát thực tế.

Trong báo cáo này, chúng tôi trình bày các phương pháp đã áp dụng khi tham gia ADC2025, 
bao gồm mô hình hóa bài toán, xây dựng khung quy trình tiền xử lý dữ liệu, tiếp cận
bài toán theo hai hướng: không sử dụng học máy (non-ML) và học máy (ML).