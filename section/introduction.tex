\section{Giới thiệu}

Protein được coi là những cỗ máy phân tử đảm nhiệm hầu hết các chức năng thiết yếu của sự sống,
từ xúc tác các phản ứng sinh hóa, vận chuyển vật chất, truyền tín hiệu đến việc tạo nên cấu trúc tế bào.
Trong những thập kỷ gần đây, sự bùng nổ của công nghệ giải trình thứ tự gen thế hệ mới (Next-Generation Sequencing - NGS)
đã tạo ra một lượng dữ liệu khổng lộ về trình tự protein. Các cơ sở dữ liệu như UniProt hiện chứa hàng 
trăm triệu chuỗi protein từ đa dạng các loài sinh vật.
\\
Tuy nhiên, tốc độ xác định chức năng của các protein này bằng thực nghiệm 
(như tinh thể học tia X hay cộng hưởng từ hạt nhân) lại chậm hơn rất nhiều do chi phí cao 
và quy trình phức tạp. Sự chênh lệch ngày càng lớn giữa lượng dữ liệu trình tự đã biết và 
lượng dữ liệu chức năng đã được xác thực tạo nên một thách thức lớn trong sinh học tính toán, 
được gọi là "Khoảng trống chú giải" (The Annotation Gap). Việc phát triển các thuật toán máy 
tính để dự đoán tự động chức năng protein (Automated Function Prediction - AFP) trở thành một 
nhu cầu cấp thiết để lấp đầy khoảng trống này.
\\
Để thúc đẩy sự phát triển của các thuật toán AFP, cuộc thi CAFA 
(Critical Assessment of Functional Annotation) đã được tổ chức như một thử thách toàn cầu, 
nơi các nhà nghiên cứu tranh tài để dự đoán chức năng của một tập hợp các protein chưa được 
chú giải. CAFA 6 là phiên bản mới nhất của chuỗi thử thách này, đặt ra yêu cầu cao hơn về độ 
chính xác và khả năng tổng quát hóa của mô hình trên các loài sinh vật khác nhau.
\\
Trong báo cáo này, chúng tôi trình bày các phương pháp đã áp dụng khi tham gia CAFA 6,
bao gồm mô hình hóa bài toán, xây dựng khung quy trình tiền xử lý dữ liệu, tiếp cận
bài toán theo ba hướng: Không học máy (non-ML) ,học máy (ML) và học sâu (DL).