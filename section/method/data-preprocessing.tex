\subsection{Tiền xử lý dữ liệu}

Dữ liệu thô ban đầu là các chuỗi trình tự protein (dạng văn bản) trong tệp FASTA. 
Để đưa vào các mô hình học máy, chúng tôi đã thử nghiệm và triển khai hai phương 
pháp tiếp cận tiền xử lý dữ liệu riêng biệt nhằm khai thác tối đa thông tin từ 
trình tự:

\begin{itemize}
  \item \textbf{Phương pháp 1: Trích xuất đặc trưng thủ công (Manual Feature Engineering)} 
  - Dựa trên kiến thức sinh học thống kê.
  \item \textbf{Phương pháp 2: Học biểu diễn ngữ nghĩa (Semantic Representation Learning)} 
  - Dựa trên mô hình ngôn ngữ lớn (ESM-2).
\end{itemize}

\subsubsection{Phương pháp 1: Trích xuất đặc trưng thủ công}\mbox{}

Ở phương pháp này, chúng tôi coi chuỗi protein là một tập hợp các thành phần hóa học và sử dụng công thức 
thống kê để biến đổi chuỗi thành vector số học. Phương pháp này tập trung vào đặc điểm thành phần (composition) thay vì thứ tự sắp xếp.

\begin{enumerate}[label=\alph*.]
\item Xử lí dữ liệu thô
  \begin{itemize}
    \item Đọc dữ liệu: sử dụng thư viện \texttt{Biopython} để phân tích cú pháp tệp FASTA.
    \item Làm sạch ID: Tách phân tiêu đề (Header) của FASTA để lấy mã định danh chuẩn (EntrID).
  \end{itemize}
\item Kỹ thuật tạo đặc trưng (Feature Engineering)\\
  Chúng tôi xây dựng vector đặc trưng $X_{manual}$ dựa trên các thuộc tính lý hóa của protein:
  \begin{itemize}
    \item \texttt{Thành phần Axit Amin (Amino Acid Composition-AAC)}: tính toán tần suất xuất hiện của 20 loại axit amin chuẩn 
    (A, R, N, D, C, Q, E, G, H, I, L, K, M, F, P, S, T, W, Y, V) trong mỗi chuỗi. Với công thức:
    \\  \begin{equation}
        AAC_i = \frac{\text{Count}(AA_i)}{\text{Length}(\text{Sequence})}
        \end{equation}
    \item \texttt{Độ dài chuỗi (Sequence Length)}: Thêm đặc trưng về tổng số lượng axit amin. 
    Độ dài chuỗi thường tương quan với độ phức tạp của chức năng protein.
  \end{itemize}

  Kết quả là mỗi protein được biểu diễn bằng một vector chiều thấp (Low-dimensional vector) với kích thước 
$D \approx 21-25$.
\end{enumerate}

\subsubsection{Phương pháp 2: Học biểu diễn ngữ nghĩa (Semantic Representation Learning)}\mbox{}

Ở phương pháp này, chúng tôi coi chuỗi protein là một "ngôn ngữ" và sử dụng mô hình học sâu 
đã được huấn luyện trước (Pre-trained Model) để trích xuất các đặc trưng ngữ nghĩa tiềm ẩn. 
Phương pháp này nắm bắt được ngữ cảnh, thứ tự và cấu trúc 3D của protein.

\begin{enumerate}[label=\alph*.]
\item Mô hình nền tảng \\
  Chúng tôi sử dụng mô hình \texttt{ESM-2} (Evolutionary Scale Modeling) phiên bản 650 triệu tham số 
  \texttt{(esm2\_t33\_650M\_UR50D)}. Đây là mô hình Transformer tiên tiến nhất hiện nay cho dữ liệu protein.
\item Quy trình embedding\\
  \begin{itemize}[label=-]
    \item \texttt{Bước 1: Mã hóa (tokenization)} - Mỗi axit amin được chuyển đổi thành token số nguyên. 
   Các token đặc biệt <cls> và <eos> được thêm vào để đánh dấu bắt đầu và kết thúc chuỗi.
    \item \texttt{Bước 2: Suy luận (inference)} - Chuỗi token được đưa qua 33 lớp Transformer của \texttt{ESM-2}. Tại đây, mô hình 
   sử dụng cơ chế Self-Attention để học mối quan hệ phức tạp giữa các axit amin, bất kể khoảng cách 
   của chúng trong chuỗi.
    \item \texttt{Bước 3: Gộp (pooling)} - Chúng tôi lấy vector trạng thái ẩn (hidden states) tại lớp cuối cùng 
   và áp dụng kỹ thuật Mean Pooling (lấy trung bình) dọc theo chiều dài chuỗi để thu được một 
   vector đại diện duy nhất cho toàn bộ protein. 
  \end{itemize}

Kết quả là mỗi protein được biểu diễn bằng một vector với kích thước D = 1280 (High-dimensional vector). 
Vector này chứa đựng thông tin phong phú về cấu trúc và chức năng của protein.

\end{enumerate}
