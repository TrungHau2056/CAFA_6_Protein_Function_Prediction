\subsection{Tiếp cận theo hướng không dùng học máy}
\subsubsection{Naive Prior Baseline: Frequency-based GO Term Prediction} \mbox{}
\label{sec:baseline_naive}
Để thiết lập một mốc tham chiếu (baseline) đơn giản, chúng tôi xây dựng 
một phương pháp dự đoán dựa hoàn toàn trên thống kê tần suất nhãn trong tập 
huấn luyện, không sử dụng bất kỳ đặc trưng nào từ chuỗi protein hay embedding. 
Ý tưởng cốt lõi là khai thác \textit{prior distribution} của các nhãn 
Gene Ontology (GO): các GO term xuất hiện càng thường xuyên trong dữ liệu huấn 
luyện được giả định có xác suất cao hơn để xuất hiện ở các protein chưa biết nhãn.

\paragraph{Ước lượng phân phối nhãn theo từng phân hệ (aspect).}
Gọi $a \in \{P, C, F\}$ lần lượt tương ứng với \textit{Biological Process}, \textit{Cellular Component} và \textit{Molecular Function}. Từ tập huấn luyện, ta đếm số lần xuất hiện của mỗi GO term $t$ trong aspect $a$:
\begin{equation}
c_a(t) = \#\{(\cdot, t)\mid \text{aspect}=a\}, 
\end{equation}
và tổng số annotation trong aspect:
\begin{equation}
N_a = \sum_{t} c_a(t).
\end{equation}
Xác suất tiên nghiệm của term $t$ trong aspect $a$ được ước lượng bằng tần suất chuẩn hoá:
\begin{equation}
p_a(t)=\frac{c_a(t)}{N_a}.
\end{equation}

\paragraph{Chọn top-$k$ GO term phổ biến nhất.}
Với mỗi aspect $a$, ta chọn tập $\mathcal{T}^{(k)}_a$ gồm $k$ term có $c_a(t)$ lớn nhất:
\begin{equation}
\mathcal{T}^{(k)}_a = \text{TopK}_{t}\big(c_a(t)\big).
\end{equation}

\paragraph{Inference}
Với mỗi protein trong tập test $x$, baseline này không phân biệt giữa các protein mà xuất ra cùng một tập dự đoán cho mọi $x$. Cụ thể, với mỗi aspect $a$, ta dự đoán tất cả các term trong $\mathcal{T}^{(k)}_a$ với điểm tin cậy bằng xác suất tiên nghiệm $p_a(t)$:
\begin{equation}
\hat{s}(x,t)=
\begin{cases}
p_a(t), & \text{nếu }\, t \in \mathcal{T}^{(k)}_a,\\
\text{không xuất ra}, & \text{ngược lại}.
\end{cases}
\end{equation}

\paragraph{Vai trò và hạn chế.}
Ưu điểm của baseline này là đơn giản, dễ tái lập và gần như không tốn chi phí tính toán. Tuy nhiên, do dự đoán không phụ thuộc vào đặc trưng protein, phương pháp không thể cá nhân hoá dự đoán theo chuỗi và chỉ phù hợp như một mốc tham chiếu để so sánh với các phương pháp học có điều kiện theo protein.

\subsubsection{BLAST - }